\documentclass[11pt]{article}
%Increase the text height
\addtolength{\voffset}{-62pt}
\addtolength{\textheight}{62pt}
\usepackage{amsmath}
\usepackage{amsmath}
\usepackage{amssymb}
\newcommand{\R}{\mathbb{R}}
\newcommand{\C}{\mathbb{C}}

%Increase the text width
\addtolength{\hoffset}{-22pt}
\addtolength{\oddsidemargin}{-32pt}
\addtolength{\marginparsep}{-11pt}
\addtolength{\marginparwidth}{-45pt}
\addtolength{\textwidth}{110pt}

\begin{document}
\pagestyle{myheadings}
\markright{\sc 230201057 Furkan Emre YILMAZ%
\hfill Math 144 HW04 /p.}

\paragraph{Q1.} Let
$
V= \{ (x_1,x_2) | x_1,x_2 \in  \R \}
$
and let
\paragraph{} $\oplus$ for $\vec{u},\vec{v} \in V$ define 
$
\vec{u} \oplus \vec{v}=(v_1,v_2) \oplus (v_1,v_2)=(v_1+v_1-2,v_2+v_2+1)
$
\paragraph{}$\odot:$ for $\alpha \in \R, \vec{v} \in V$ define
$
\alpha \odot \vec{v}=\alpha \odot (v_1,v_2)=(\alpha v_2 + \alpha -1,\alpha v_1 -2\alpha+2)
$
\\
\\
Is V a vector space over $\R$ with the above operations?

\paragraph{Solution 1.}I want to approach this problem using vector space properties so, suppose that
\begin{eqnarray*}
(-1) \odot \vec{u}&=&-\vec{u} \text{(Proof is given in the last question answers)}
\\ \vec{u} \oplus (-\vec{u})&=&\vec{0}
\\ \vec{u} \oplus \vec{0} &=& \vec{u}
\end{eqnarray*}
\paragraph{}so lets continue chosing a vector in the set such that $x_1,x_2 \in \R$
\begin{eqnarray*}
\vec{u}&=&(1,2)
\\ -1 \odot (1,2) &=& ((-1)(2)+(-1)-1,(-1)(1)-2(-1)+2)
\\ -\vec{u} &=& (-4,3)
\\ (1,2) \oplus (-4,3) &=& (1+(-4)-2,2+3+1)
\\ \vec{0}&=&(-5,6)
\\ (1,2) \oplus (-5,6)&=&(1+(-5)-2,2+6+1)
\\ &=& (-6,9)
\\ \vec{u} &\neq& (-6,9)
\end{eqnarray*}
\paragraph{}so the equation doesn't satisfy our expected result. We can say that it is not a vector space over the $\R$


\paragraph{Q2.}Let $V=\{ (x_1,x_2) | x_1,x_2 \in \R, x_1x_2 \neq 0 \}$ and let 
\paragraph{}$\oplus:$ for $\vec{u},\vec{v} \in V$ define 
$
\vec{u} \oplus \vec{v}=(u_1,u_2) \oplus (v_1,v_2)=(u_1+v_1,u_2+v_2)
$
\paragraph{}$\odot:$ for $\alpha \in \R$ define 
$
\alpha \odot \vec{v}=\alpha \odot (v_1,v_2)=(\alpha v_1,\alpha v_2)
$
\\
\\
Is V a vector space over $\R$ with the above operations?

\paragraph{Solution 2.}To be a vector space means that all the vector properties is satisfied so lets start our observations vital part of vector spaces
\\
\\
vector addition denoted by $\oplus$
\paragraph{}$\oplus : V \times V \rightarrow V (1)$
\\
\\
scalar vector multiplication denoted by $\odot$
\paragraph{}$\odot : \R \times V \rightarrow V (2)$
so lets given an example and show it doesn't satify equation 1
\begin{eqnarray*}
(v_1,v_2) \oplus (-v_1,-v_2) &=& (0,0) 
\\x_1 x_2 &=& 0
\\(0,0) &\notin & V
\\(1,3) \oplus  (-1,4) & = &(0,7)
\\x_1 x_2 &=& 0
\\(0,7) & \notin & V
\end{eqnarray*}
so $x_1,x_2 \in \R$ but it doesn't satisfy property 1 $\oplus : V \times V \rightarrow V$
\paragraph{Q3.}Is the set of all invertible two by two matrices a vector space?
\paragraph{Solution 3.}Lets start to work pointing out our vector operations.Question says invertible two by two matrices so for invertible two by two matrices we have
\paragraph{}V=$\{$For each $A_{2x2}$ we have $B_{2x2}$ such that AxB=I$\}$
\paragraph{}
$\oplus :
\begin{pmatrix}
x_1&x_2\\
x_3&x_4
\end{pmatrix}
\oplus
\begin{pmatrix}
x_1&x_2\\
x_3&x_4
\end{pmatrix}
=
\begin{pmatrix}
x_1+x_1&x_2+x_2\\
x_3+x_3&x_4+x_4
\end{pmatrix}
x_1,x_2,x_3,x_4 \in \C
$ 
\paragraph{}
$
\odot :
\alpha
\odot
\begin{pmatrix}
x_1&x_2\\
x_3&x_4
\end{pmatrix}
=
\begin{pmatrix}
\alpha x_1& \alpha x_2\\
\alpha x_3& \alpha x_4
\end{pmatrix}
\alpha \in \C
$
$
x_1,x_2,x_3,x_4 \in \C
$
\paragraph{}and we know that our vector space must statisfy all vector properties.Let's continue our observation this
\\
vector addition denoted by $\oplus$
\paragraph{}$\oplus : V \times V \rightarrow V (1)$
\\
\\
scalar vector multiplication denoted by $\odot$
\paragraph{}$\odot : \C \times V \rightarrow V (2)$
\\
\\
so lets given an example and show it doesn't satify property 1 again
\begin{eqnarray*}
\begin{pmatrix}
1 & 0\\
0 & 1
\end{pmatrix}
\oplus
\begin{pmatrix}
-1 & 0\\
0 & -1
\end{pmatrix}
&=&
\begin{pmatrix}
0 & 0\\
0 & 0
\end{pmatrix}
\\ \begin{pmatrix}
0 & 0\\
0 & 0
\end{pmatrix} & \notin & V
\end{eqnarray*}
we can clearly see that addtion operation takes place invertible two by two matrices but result of this operation isn't element of our set so \textbf{ we can say that invertible two by two matrices don't struct a vector space.} We can give lots of example doesn't satisfy this property.

\paragraph{Q4.}Compute
\paragraph{}1.
$
\left\{
(-1) \odot
\left[
\begin{pmatrix}
8\\
1
\end{pmatrix}
\oplus
\begin{pmatrix}
0\\
1
\end{pmatrix}
\right]
\right\}
\oplus
\left\{
3
\odot
\left[
\begin{pmatrix}
2\\
0
\end{pmatrix}
\oplus
\begin{pmatrix}
0\\
0
\end{pmatrix}
\right]
\right\}
$
\paragraph{}2.
$
\left\{
\left[
0
\odot
\begin{pmatrix}
2\\
3
\end{pmatrix}
\right]
\oplus
\begin{pmatrix}
4\\
-1
\end{pmatrix}
\right\}
\oplus
\left\{
\left[
3
\odot
\begin{pmatrix}
1\\
2
\end{pmatrix}
\right]
\oplus
\left[
2
\odot
\begin{pmatrix}
9\\
1
\end{pmatrix}
\right]
\right\}
$
\paragraph{}3.
$
\left\{
0
\odot
\left[
\begin{pmatrix}
2\\
3
\end{pmatrix}
\oplus
\begin{pmatrix}
4\\
-1
\end{pmatrix}
\right]
\right\}
\oplus
\left\{
7
\odot
\left[
\begin{pmatrix}
1\\
2
\end{pmatrix}
\oplus
\begin{pmatrix}
9\\
1
\end{pmatrix}
\right]
\right\}
$

\paragraph{Solution 4.}Let start to solve equations step by step
\paragraph{}1.
$
\underbrace{
\underbrace{
\left\{
(-1) \odot
\underbrace{
\left[
\begin{pmatrix}
8\\
1
\end{pmatrix}
\oplus
\begin{pmatrix}
0\\
1
\end{pmatrix}
\right]
}_\text{Step 1.}
\right\}
}_\text{Step 2.}
\oplus
\underbrace{
\left\{
3
\odot
\underbrace{
\left[
\begin{pmatrix}
2\\
0
\end{pmatrix}
\oplus
\begin{pmatrix}
0\\
0
\end{pmatrix}
\right]
}_\text{Step 3.}
\right\}
}_\text{Step 4.}
}_\text{Step 5.}
$
\paragraph{Step 1.}
\begin{eqnarray*}
\begin{pmatrix}
8\\
1
\end{pmatrix}
\oplus
\begin{pmatrix}
0\\
1
\end{pmatrix}
&=&
\begin{pmatrix}
8+0-4\\
1+1-3
\end{pmatrix}
\\
&=&
\begin{pmatrix}
4\\
-1
\end{pmatrix}
\end{eqnarray*}
\paragraph{Step 2.}
\begin{eqnarray*}
-1
\odot
\begin{pmatrix}
4\\
-1
\end{pmatrix}
&=&
\begin{pmatrix}
(-1)4-4(-1)+4\\
(-1)(-1)-3(-1)+3
\end{pmatrix}
\\&=&
\begin{pmatrix}
4\\
7
\end{pmatrix}
\end{eqnarray*}
\paragraph{Step 3.}
\begin{eqnarray*}
\begin{pmatrix}
2\\
0
\end{pmatrix}
\oplus
\begin{pmatrix}
0\\
0
\end{pmatrix}
&=&
\begin{pmatrix}
2+0-4\\
0+0-3
\end{pmatrix}
\\ &=&
\begin{pmatrix}
-2\\
-3
\end{pmatrix}
\end{eqnarray*}
\paragraph{Step 4.}
\begin{eqnarray*}
3
\odot
\begin{pmatrix}
-2\\
-3
\end{pmatrix}
&=&
\begin{pmatrix}
3(-2)-4(3)+4\\
3(-3)-3(3)+3
\end{pmatrix}
\\&=&
\begin{pmatrix}
-14\\
-15
\end{pmatrix}
\end{eqnarray*}
\paragraph{Step 5.}
\begin{eqnarray*}
\begin{pmatrix}
4\\
7
\end{pmatrix}
\oplus
\begin{pmatrix}
-14\\
-15
\end{pmatrix}
&=&
\begin{pmatrix}
4+(-14)-4\\
7+(-15)-3
\end{pmatrix}
\\&=&
\begin{pmatrix}
-14\\
-11
\end{pmatrix}
\end{eqnarray*}
\paragraph{}\textbf{ Step 5 is our final step at it is our result as well.}


\paragraph{}2.
$
\underbrace{
\underbrace{
\left\{
\underbrace{
\left[
0
\odot
\begin{pmatrix}
2\\
3
\end{pmatrix}
\right]
}_\text{Step 1.}
\oplus
\begin{pmatrix}
4\\
-1
\end{pmatrix}
\right\}
}_\text{Step 2.}
\oplus
\underbrace{
\left\{
\underbrace{
\left[
3
\odot
\begin{pmatrix}
1\\
2
\end{pmatrix}
\right]
}_\text{Step 3.}
\oplus
\underbrace{
\left[
2
\odot
\begin{pmatrix}
9\\
1
\end{pmatrix}
\right]
}_\text{Step 4.}
\right\}
}_\text{Step 5.}
}_\text{Step 6.}
$
\paragraph{Step 1.}
\begin{eqnarray*}
0
\odot
\begin{pmatrix}
2\\
3
\end{pmatrix}
&=&
\begin{pmatrix}
(0)2-4(0)+4\\
(0)3-3(0)+3
\end{pmatrix}
\\&=&
\begin{pmatrix}
4\\
3
\end{pmatrix}
\end{eqnarray*}
\paragraph{Step 2.}
\begin{eqnarray*}
\begin{pmatrix}
4\\
3
\end{pmatrix}
\oplus
\begin{pmatrix}
4\\
-1
\end{pmatrix}
&=&
\begin{pmatrix}
4+4-4\\
3+(-1)-3
\end{pmatrix}
\\&=&
\begin{pmatrix}
4\\
-1
\end{pmatrix}
\end{eqnarray*}
\paragraph{Step 3.}
\begin{eqnarray*}
3
\odot
\begin{pmatrix}
1\\
2
\end{pmatrix}
&=&
\begin{pmatrix}
(3)1-4(3)+4\\
(3)2-3(3)+3
\end{pmatrix}
\\&=&
\begin{pmatrix}
-5\\
0
\end{pmatrix}
\end{eqnarray*}
\paragraph{Step 4.}
\begin{eqnarray*}
2
\odot
\begin{pmatrix}
9\\
1
\end{pmatrix}
&=&
\begin{pmatrix}
(2)9-4(2)+4\\
(2)1-3(2)+3
\end{pmatrix}
\\&=&
\begin{pmatrix}
14\\ 
-1
\end{pmatrix}
\end{eqnarray*}
\paragraph{Step 5.}
\begin{eqnarray*}
\begin{pmatrix}
-5\\
0
\end{pmatrix}
\oplus
\begin{pmatrix}
14\\
-1
\end{pmatrix}
&=&
\begin{pmatrix}
-5+14-4\\
0+(-1)-3
\end{pmatrix}
\\&=&
\begin{pmatrix}
5\\
-4
\end{pmatrix}
\end{eqnarray*}
\paragraph{Step 6.}
\begin{eqnarray*}
\begin{pmatrix}
4\\
-1
\end{pmatrix}
\oplus
\begin{pmatrix}
5\\
-4
\end{pmatrix}
&=&
\begin{pmatrix}
4+5-4\\
(-1)+(-4)-3
\end{pmatrix}
\\&=&
\begin{pmatrix}
5\\
-8
\end{pmatrix}
\end{eqnarray*}
\paragraph{}\textbf{Step 6 is our result for the whole operations and system.}
\paragraph{}3.
$
\underbrace{
\underbrace{
\left\{
0 \odot
\underbrace{
\left[
\begin{pmatrix}
2\\
3
\end{pmatrix}
\oplus
\begin{pmatrix}
4\\
-1
\end{pmatrix}
\right]
}_\text{Step 1.}
\right\}
}_\text{Step 2.}
\oplus
\underbrace{
\left\{
7
\odot
\underbrace{
\left[
\begin{pmatrix}
1\\
2
\end{pmatrix}
\oplus
\begin{pmatrix}
9\\
1
\end{pmatrix}
\right]
}_\text{Step 3.}
\right\}
}_\text{Step 4.}
}_\text{Step 5.}
$
\paragraph{Step 1.}
\begin{eqnarray*}
\begin{pmatrix}
2\\
3
\end{pmatrix}
\oplus
\begin{pmatrix}
4\\
-1
\end{pmatrix}
&=&
\begin{pmatrix}
2+4-4\\
3+(-1)-3
\end{pmatrix}
\\&=&
\begin{pmatrix}
2\\
-1
\end{pmatrix}
\end{eqnarray*}
\paragraph{Step 2.}
\begin{eqnarray*}
0
\odot
\begin{pmatrix}
2\\
-1
\end{pmatrix}
&=&
\begin{pmatrix}
(0)2-4(0)+4\\
(0)(-1)-3(0)+3
\end{pmatrix}
\\&=&
\begin{pmatrix}
4\\
3
\end{pmatrix}
\end{eqnarray*}
\paragraph{Step 3.}
\begin{eqnarray*}
\begin{pmatrix}
1\\
2
\end{pmatrix}
\oplus
\begin{pmatrix}
9\\
1
\end{pmatrix}
&=&
\begin{pmatrix}
1+9-4\\
2+1-3
\end{pmatrix}
\\&=&
\begin{pmatrix}
6\\
0
\end{pmatrix}
\end{eqnarray*}
\paragraph{Step 4.}
\begin{eqnarray*}
7
\odot
\begin{pmatrix}
6\\
0
\end{pmatrix}
&=&
\begin{pmatrix}
(7)6-4(7)+4\\
(7)0-3(0)+3
\end{pmatrix}
\\&=&
\begin{pmatrix}
18\\
3
\end{pmatrix}
\end{eqnarray*}
\paragraph{Step 5.}
\begin{eqnarray*}
\begin{pmatrix}
4\\
3
\end{pmatrix}
\oplus
\begin{pmatrix}
18\\
3
\end{pmatrix}
&=&
\begin{pmatrix}
4+18-4\\
3+3-3
\end{pmatrix}
\\&=&
\begin{pmatrix}
18\\
3
\end{pmatrix}
\end{eqnarray*}
\paragraph{}\textbf{Last step our result for whole operations}

\paragraph{Q5.}Find the additive inverse of
\paragraph{}1.
$
\begin{pmatrix}
0\\
0
\end{pmatrix}
$
\paragraph{}2.
$
\begin{pmatrix}
8\\
1
\end{pmatrix}
$
\paragraph{}3.
$
(-4)
\odot
\begin{pmatrix}
4\\
3
\end{pmatrix}
$
\paragraph{Solution 5.}To find additive inverses I want to use Theorem AISM
\paragraph{Theorem AISM}Additive Inverses from Scalar Multiplication
\paragraph{}\textit{Suppose that V is a vector space and \textbf{u} $\in$ V.Then \textbf{-u}=(-1)\textbf{u}.}
\paragraph{Proof}
\begin{eqnarray*}
-u&=&-u+\vec{0}
\\&=& -u + \vec{0}u
\\&=& -u +(1+(-1))u
\\&=& -u +(1u+(-1)u)
\\&=& -u +(u + (-1)u)
\\&=& (-u+u) +(-1)u
\\&=& \vec{0}+(-1)u
\\&=&(-1)u
\end{eqnarray*}

\paragraph{}Using this theorem we can easily find additive inverse of the vectors let's start working with first one
\paragraph{1.} 
\begin{eqnarray*}
\vec{u}
&=&
\begin{pmatrix}
0\\
0
\end{pmatrix}
\\
-\vec{u}
&=&
-1
\odot
\begin{pmatrix}
0\\
0
\end{pmatrix}
\\&=&
\begin{pmatrix}
(-1)0-4(-1)+4\\
(-1)0-3(-1)+3
\end{pmatrix}
=
\underbrace{
\begin{pmatrix}
8\\
6
\end{pmatrix}
}_\text{-$\vec{u}$}
\end{eqnarray*}
\paragraph{}Lets test our result using vector properties $\vec{u}-\vec{u}=\vec{0}$
\begin{eqnarray*}
\begin{pmatrix}
0\\
0
\end{pmatrix}
\oplus
\begin{pmatrix}
8\\
6
\end{pmatrix}
=
\begin{pmatrix}
0+8-4\\
0+6-3
\end{pmatrix}
=
\begin{pmatrix}
4\\
3
\end{pmatrix}
\\
\vec{0}
=
\begin{pmatrix}
4\\
3
\end{pmatrix}
\end{eqnarray*}
\paragraph{}As you can see it satisfy property of vector space
\paragraph{2.}
\begin{eqnarray*}
\vec{u}
&=&
\begin{pmatrix}
8\\
1
\end{pmatrix}
\\
-\vec{u}
&=&
(-1)
\odot
\begin{pmatrix}
8\\
1
\end{pmatrix}
\\
&=&
\begin{pmatrix}
(-1)8+(-4)(-1)+4\\
(-1)1+(-3)(-1)+3
\end{pmatrix}
=
\underbrace{
\begin{pmatrix}
0\\
5
\end{pmatrix}
}_\text{- $\vec{u}$ }
\end{eqnarray*}
\paragraph{}Lets test our result using vector properties $\vec{u}-\vec{u}=\vec{0}$
\begin{eqnarray*}
\begin{pmatrix}
8\\
1
\end{pmatrix}
\oplus
\begin{pmatrix}
0\\
5
\end{pmatrix}
=
\begin{pmatrix}
8+0-4\\
1+5-3
\end{pmatrix}
&=&
\begin{pmatrix}
4\\
3
\end{pmatrix}
\\
\vec{0}
&=&
\begin{pmatrix}
4\\
3
\end{pmatrix}
\end{eqnarray*}
\paragraph{3.}Result of this scalar multiplication is zero vector because 
$
\begin{pmatrix}
4\\
3
\end{pmatrix}
$
is zero vector for this vector space we already came this result subuqestion of 1,2 so,
\begin{eqnarray*}
(-4)
\odot
\begin{pmatrix}
4\\
3
\end{pmatrix}
=
\begin{pmatrix}
4\\
3
\end{pmatrix}
\end{eqnarray*}
\paragraph{}Additive inverse of zero vector is itself because it satisfy $\vec{u}-\vec{u}=\vec{0}$
\begin{eqnarray*}
-1
\odot
\begin{pmatrix}
4\\
3
\end{pmatrix}
=
\begin{pmatrix}
4\\
3
\end{pmatrix}
\end{eqnarray*}

\end{document}